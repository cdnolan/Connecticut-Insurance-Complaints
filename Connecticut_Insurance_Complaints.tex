% Options for packages loaded elsewhere
\PassOptionsToPackage{unicode}{hyperref}
\PassOptionsToPackage{hyphens}{url}
%
\documentclass[
]{article}
\title{Connecticut Insurance Complaints 2015-2021}
\author{Cliff Nolan}
\date{10/22/2021}

\usepackage{amsmath,amssymb}
\usepackage{lmodern}
\usepackage{iftex}
\ifPDFTeX
  \usepackage[T1]{fontenc}
  \usepackage[utf8]{inputenc}
  \usepackage{textcomp} % provide euro and other symbols
\else % if luatex or xetex
  \usepackage{unicode-math}
  \defaultfontfeatures{Scale=MatchLowercase}
  \defaultfontfeatures[\rmfamily]{Ligatures=TeX,Scale=1}
\fi
% Use upquote if available, for straight quotes in verbatim environments
\IfFileExists{upquote.sty}{\usepackage{upquote}}{}
\IfFileExists{microtype.sty}{% use microtype if available
  \usepackage[]{microtype}
  \UseMicrotypeSet[protrusion]{basicmath} % disable protrusion for tt fonts
}{}
\makeatletter
\@ifundefined{KOMAClassName}{% if non-KOMA class
  \IfFileExists{parskip.sty}{%
    \usepackage{parskip}
  }{% else
    \setlength{\parindent}{0pt}
    \setlength{\parskip}{6pt plus 2pt minus 1pt}}
}{% if KOMA class
  \KOMAoptions{parskip=half}}
\makeatother
\usepackage{xcolor}
\IfFileExists{xurl.sty}{\usepackage{xurl}}{} % add URL line breaks if available
\IfFileExists{bookmark.sty}{\usepackage{bookmark}}{\usepackage{hyperref}}
\hypersetup{
  pdftitle={Connecticut Insurance Complaints 2015-2021},
  pdfauthor={Cliff Nolan},
  hidelinks,
  pdfcreator={LaTeX via pandoc}}
\urlstyle{same} % disable monospaced font for URLs
\usepackage[margin=1in]{geometry}
\usepackage{graphicx}
\makeatletter
\def\maxwidth{\ifdim\Gin@nat@width>\linewidth\linewidth\else\Gin@nat@width\fi}
\def\maxheight{\ifdim\Gin@nat@height>\textheight\textheight\else\Gin@nat@height\fi}
\makeatother
% Scale images if necessary, so that they will not overflow the page
% margins by default, and it is still possible to overwrite the defaults
% using explicit options in \includegraphics[width, height, ...]{}
\setkeys{Gin}{width=\maxwidth,height=\maxheight,keepaspectratio}
% Set default figure placement to htbp
\makeatletter
\def\fps@figure{htbp}
\makeatother
\setlength{\emergencystretch}{3em} % prevent overfull lines
\providecommand{\tightlist}{%
  \setlength{\itemsep}{0pt}\setlength{\parskip}{0pt}}
\setcounter{secnumdepth}{-\maxdimen} % remove section numbering
\ifLuaTeX
  \usepackage{selnolig}  % disable illegal ligatures
\fi

\begin{document}
\maketitle

\hypertarget{task}{%
\subsubsection{Task}\label{task}}

This analysis seeks to identify trends in customer complaints made
against insurance companies from 2018 to 2021 in Connecticut in order to
create suggestions to improve customer satisfaction and retention.This
data was obtained from the
\href{https://data.ct.gov/Business/Insurance-Company-Complaints-Resolutions-Status-an/t64r-mt64}{Connecticut
Open Data website} on October 19, 2021

\hypertarget{narrowing-focus}{%
\subsubsection{Narrowing Focus}\label{narrowing-focus}}

The majority of complaints are alleging unfair claims practices. Of
those, the four largest sub-reasons are unsatisfactory settlement,
denial of claim, claim procedure, and claim delays.

\includegraphics{Connecticut_Insurance_Complaints_files/figure-latex/graph_reasons-1.pdf}

\hypertarget{section}{%
\subsubsection{}\label{section}}

\includegraphics{Connecticut_Insurance_Complaints_files/figure-latex/graph_subreasons-1.pdf}
When looking at the conclusions reached for these claims, we can see
that unjustified claims occur most frequently. This is true for all of
the largest sub-reason groups. In all but one (claim delays), the
second-most frequent result is that the customer is given further
information. This presents an opportunity for improvement as their
dissatisfaction is based on their perception. These customers have
complaints whether or not they were actually treated unfairly.
\includegraphics{Connecticut_Insurance_Complaints_files/figure-latex/graph_conclusions-1.pdf}

\end{document}
